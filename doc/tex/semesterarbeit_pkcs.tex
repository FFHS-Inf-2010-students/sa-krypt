\documentclass[10pt,a4paper]{article}
\usepackage{ucs}
\usepackage[utf8x]{inputenc}
\usepackage[ngerman]{babel}
\selectlanguage{ngerman}
\usepackage[T1]{fontenc}

% Math stuff
\usepackage{amsmath}
\usepackage{amsfonts}
\usepackage{amssymb}

% Farben
\usepackage[usenames,x11names,dvipsnames,rgb]{xcolor}
\definecolor{grey}{rgb}{0.4,0.4,0.4}
\definecolor{lightgrey}{rgb}{0.8,0.8,0.8}
\definecolor{ultralightgrey}{rgb}{0.96,0.96,0.96}

% Grafix
\usepackage{graphicx}

% TikZ (dot2tex etc.)
\usepackage{tikz}
\usetikzlibrary{decorations, arrows, shapes}

% Farben in Tabellen
\usepackage{colortbl}

% Lange Tabellen
\usepackage{longtable}

% Gewrappte boxen (können innerhalb f{rame}box's verwendet werden)
\usepackage{minibox}

% FloatBarrier stellt sicher, dass das Literaturverzeichnis am Ende des
% Dokuments erscheint.
\usepackage{placeins}

% Hyperref
\usepackage{hyperref}
% Hypersetup
\hypersetup{
    pdftitle = {Semesterarbeit Kryptographie - Gruppenarbeit PKCS},
    pdfauthor = {Aregger Thomas, Gutknecht Jürg, Dünki Marc, Daniel David},
    pdfsubject = {Semesterarbeit Kryptographie - Gruppenarbeit PKCS},
    pdfkeywords = {Kryptographie PKCS Semesterarbeit},
    % hidelinks
    colorlinks = true,
    linkcolor = blue,
    % urlcolor = black
    urlcolor = Blue,
    citecolor = grey
}

\urlstyle{same}

\title{Semesterarbeit Kryptographie - Gruppenarbeit  PKCS}
\author{
    Aregger Thomas \small{thomas.aregger@students.ffhs.ch}\\
    Dünki Marc \small{marc.duenki@students.ffhs.ch}\\
    Gutknecht Jürg \small{juerg.gutknecht@students.ffhs.ch}\\
    Daniel David \small{david.daniel@students.ffhs.ch}}

\date{\today}

\begin{document}

\maketitle
% \tableofcontents

\section{ASN.1}

\subsection{Verwendete Namen und Bezeichnungen}
\begin{description}
    \item[Object identifier] Eine Folge von Ganzzahlen welche ein spezifisches Objekt
        identifiziert.
\end{description}

\section{PKCS 7 - Standard zur Syntax kryptographischer Nachrichten}

\begin{description}
    \item[PKCS \#7] definiert eine generelle Syntax zur Beschreibung von Inhalten welche in
        Verbindung mit kryptographischen Verfahren stehen können, zum Beispiel Signaturen.
        Die Syntax erlaubt Rekursion, so dass beispielsweise Inhalte signiert werden
        können, welche zuvor von einer anderen Instanz signiert wurden etc.
\end{description}

Folgendes sind Beispiele von Anwendungen, welche dieser Standard adressiert:
\begin{itemize}
    \item Signieren von digitalen Nachrichten
    \item Digest (hash) von digitalen Nachrichten
    \item Authentisierung von Nachrichten (MAC)
    \item Verschlüsselung digitaler Inhalte
\end{itemize}

\subsection{Aufbau}

Es werden generell zwischen zwei Klassen von Inhalts-Typen unterschieden:
\begin{description}
    \item[base] data enthält "`plain data"', also Daten, welche keine kryptographischen
        Erweiterungen ("`enhancements"') aufweisen.
    \item[enhanced] data enthält Inhalt eines bestimmten Typs (evtl. verschlüsselt) und
        weitere kryptographische Erweiterungen.
\end{description}

Insgesamt werden vom Standard sechs Inhalts-Typen definiert, weitere könnten in der
Zukunft hinzu kommen:
\label{list:content-types}
\begin{itemize}
    \item data
    \item signed data
    \item enveloped data
    \item signed-and-enveloped data
    \item digested data
    \item encrypted data
\end{itemize}

Nur "`data"' gehört zur Klasse "`base"', alle weiteren Typen gehören zur Kategorie
"`enhanced"'. Da die Typen der enhanced Klasse selbst Inhalte anderer Typen beinhalten,
wird auch von dem sogenannten "`inner"' und "`outer"' content gesprochen. Der "`inner"'
content ist somit der vom "`outer"' content erweiterte Inhalt.

\subsection{Der Inhalt}

Der Standard exportiert schliesslich einen Typen \texttt{ContentInfo}, welcher den
entsprechenden Inhalt zu repräsentieren vermag. Eine Nachricht, ein Element resp. ein
Objekt dieses Standards weist die folgende Syntax auf:

\begin{verbatim}
ContentInfo ::= SEQUENCE {
    contentType ContentType,
    content
        [0] EXPLICIT ANY DEFINED BY contentType OPTIONAL
}
\end{verbatim}

\begin{description}
    \item[contentType] benennt den Typ des Inhaltes. Es handelt sich um einen Object
        identifier, welcher den Inhalts-Typen gemäss obiger Liste~\ref{list:content-types}
        (data, signed data etc.) enthält. Er weist die folgende Syntax auf:
        \begin{verbatim}
        ContentType ::= OBJECT IDENTIFIER
        \end{verbatim}
    \item[content] ist der Inhalt der Nachricht. Das Feld ist optional und falls es nicht
        enthalten ist, muss der gewünschte Inhalt anderweitig zur Verfügung gestellt
        werden (wird mittels \texttt{contentType} kommuniziert).
\end{description}

\subsection{Die Inhalts-Typen}

\subsubsection{Data}
Der data content type ist ein arbiträrer octet string, welcher generell keine interne
Struktur aufweist (jedoch aufweisen kann) und in Form einer ASCII Zeichenketten betrachtet
wird. Die genaue Interpretation des Inhaltes wird der Anwendung überlassen.

\subsubsection{Signed-data}
Der signed-data content type besteht aus dem signierten Inhalt eines beliebigen Typs sowie
verschlüsselter Digests des Inhaltes, welche von einer beliebigen Anzahl Instanzen
signiert wurden.

Die Syntax des signet-data content type ist folgendermassen gegeben \cite[S.10]{pkcs7}:
\begin{verbatim}
SignedData ::= SEQUENCE {
    version Version,
    digestAlgorithms DigestAlgorithmIdentifiers,
    contentInfo ContentInfo,
    certificates [0] IMPLICIT ExtendedCertificatesAndCertificates
        OPTIONAL,
    crls [1] IMPLICIT CertificateRevocationLists OPTIONAL,
    signerInfos SignerInfos
}
DigestAlgorithmIdentifiers ::= SET OF DigestAlgorithmIdentifier
SignerInfos ::= SET OF SignerInfo
\end{verbatim}

Der Prozess, wie signierter Inhalt erzeugt wird, wird folgendermassen festgehalten
\cite[S.10]{pkcs7}:
\begin{enumerate}
    \item Pro jede signierende Instanz wird der Message Digest des Inhaltes gemäss dem
        Algorithmus der signierenden Instanz erstellt.
    \item Pro signierende Instanz wird mit dessen privatem Schlüssel jeder Message Digest
        dieser Instanz und dessen zugehörigen Angaben verschlüsselt.
    \item Pro signierende Instanz werden der verschlüsselte Message Digest und andere
        Instanz-spezifische Informationen in einem \texttt{SignerInfo} Objekt abgelegt.
        Die Zertifikate und certificate-revocation Listen werden in diesem Schritt
        ermittelt.
    \item Sämtliche Message-Digest Algorithmen und \texttt{SignerInfo} Objekte für alle
        signierenden Instanzen werden mit dem Inhalt im \texttt{SignedDataValue} Objekt
        abgelegt.
\end{enumerate}

\subsubsection{Enveloped}
Der enveloped-data content type beinhaltet verschlüsselte Daten sowie die verschlüsselten
Schlüssel für eine beliebige Anzahl Empfänger, womit der Inhalt wieder entschlüsselt
werden kann. Die Idee ist ein sogenanntes "`Digital Envelope"', dadurch können die
Vorteile des Public Key Algorithmus mit den Vorzügen der symmetrischen Verschlüsselung
genutzt werden (hybrid).

Das Element enthält neben einer Version und den verschlüsselten Inhalten eine Menge von
Empfänger Angaben. Darin werden der verschlüsselte Schlüssel sowie die Angaben über den
zugrundeliegenden Algorithmus hinterlegt:
\begin{verbatim}
EnvelopedData ::= SEQUENCE {
    version Version,
    recipientInfos RecipientInfos,
    encryptedContentInfo EncryptedContentInfo
}
RecipientInfos ::= SET OF RecipientInfo
EncryptedContentInfo ::= SEQUENCE {
    contentType ContentType,
    contentEncryptionAlgorithm
    ContentEncryptionAlgorithmIdentifier,
    encryptedContent
    [0] IMPLICIT EncryptedContent OPTIONAL
}
EncryptedContent ::= OCTET STRING
\end{verbatim}
\label{syntax:enveloped}

\subsubsection{Signed and enveloped}
Der signed and enveloped content type kombiniert quasi die signed und die enveloped Typen.
Im Besonderen wird der genaue Prozess zur Erstellung des entsprechenden Inhaltes genannt
\cite[S.22]{pkcs7} und der genaue Aufbau des Typs beschrieben \cite[S.23]{pkcs7}.

\subsubsection{Digested}
Der digested content type beinhaltet einen Inhalt beliebigen Typs sowie einen Message
Digest dazu. Dies dient grundsätzlich dazu, die Integrität des Inhaltes zu gewährleisten.
Dieser Inhalt wird daher typischerweise in einen enveloped content type integriert.

\subsubsection{Encrypted}
Der encrypted content type Inhalt enthält selbst keine Empfänger oder Schlüssel. Der Typ
ist eher dazu gedacht, für lokale Verschlüsselung verwendet zu werden. Der Typ beinhaltet
lediglich eine Version und den verschlüsselten Inhalt wie er Bestandteil des enveloped
content type ist (Syntax \ref{syntax:enveloped}, \texttt{EncryptedContentInfo}).

\section{PKCS 9 - Ausgewählte Objektklassen und Attribute}
In PKCS\#9 werden einige Objektklassen und Attribute behandelt, welche Bestandteile
anderer PKCS Dokumente sind und von unterschiedlichen Dokumenten gleichermassen
referenziert werden.

Der Standard führt die folgenden zwei neuen Objekt-Klassen und deren zugehörigen Attribute
ein:
\begin{itemize}
    \item \texttt{pkcsEntity}
    \item \texttt{naturalPerson}
\end{itemize}

\subsection{pkcsEntity}
Die \texttt{pcksEntity} Objekt-Klasse ist dazu gedacht, Attribute von beliebigen PKCS
Entitäten zu beherbergen. Sie wurde für die Verwendung von LDAP basierten
Verzeichnis-Diensten entwickelt.

Die Syntax ist folgendermassen gegeben: 
\begin{verbatim}
pkcsEntity OBJECT-CLASS ::= {
    SUBCLASS OF { top }
    KIND auxiliary
    MAY CONTAIN { PKCSEntityAttributeSet }
    ID pkcs-9-oc-pkcsEntity
}
\end{verbatim}

Die Klasse sieht eine ID sowie ein optionales Attribut vor. Die folgenden Attribute werden
allesamt dafür verwendet, die zugehörigen Informationen in einem Verzeichnis-Dienst
abzulegen.
\begin{description}
    \item[pKCS7PDU] Die in PKCS\#7 definierten geschützten Daten (enveloped, signed etc.)
        werden mit diesem Attribut verwendet.
    \item[userPKCS12] In PKCS\#12 wird ein Format für den Austausch von Angaben über die
        persönliche Identität definiert. Dieses Attribut kann hierfür verwendet werden.
    \item[pKCS15Token] In PKCS\#15 wird ein Format für kryptographische Tokens definiert,
        welche in diesem Attribut abgelegt werden können.
    \item[encryptedPrivateKeyInfo] PCKS\#8 definiert ein Format für verschlüsselte private
        Schlüssel, welche in diesem Attribut enthalten sein können.
\end{description}

\subsection{naturalPerson}
Die Objekt-Klasse \texttt{naturalPerson} wurde wie die Klasse \texttt{pkcsEntity} für die
Verwendung in Verzeichnis-Diensten erstellt. \texttt{naturalPerson} ist dafür gedacht,
Attribute von natürlichen Personen (Menschen) zu beherbergen.

Die Syntax und der Aufbau ähneln stark derjenigen der \texttt{pkcsEntity}:
\begin{verbatim}
naturalPerson OBJECT-CLASS ::= {
    SUBCLASS OF { top }
    KIND auxilary
    MAY CONTAIN { NaturalPersonAttributeSet }
    ID pkcs-9-oc-naturalPerson
}
\end{verbatim}

Für \texttt{naturalPerson} sind die folgenden Attribute definiert:
\begin{description}
    \item[emailAddress] spezifiziert eine oder mehrere E-Mail Adressen in Form einer
        unstrukturierten ASCII Zeichenkette. Es liegt an der Anwendung, die Adressen zu
        interpretieren. Speziell an diesem Element ist die \texttt{EQUALITY MATCHING RULE}
        als \texttt{pkcs9CaseIgnoreMatch}, welche besagt, dass falls zwei E-Mail Adressen
        miteinander verglichen werden, wird die Gross-Kleinschreibung ignoriert.
    \item[unstructuredName] spezifiziert den oder die Namen eines Subjektes als
        unstrukturierte ASCII Zeichenkette. Ein unstrukturierter Name kann mehrere
        Attribut-Werte enthalten, es liegt auch hier an der Anwendung, den Namen zu
        interpretieren. Wie bei der E-Mail Adresse wird beim Vergleich zweier
        unstrukturierter Adressen die Gross-Kleinschreibung ignoriert.
    \item[unstructuredAddress] nennt die Adresse des Subjektes. Auch hierbei handelt es
        sich um vollkommen unstrukturierten Inhalt. Auch hier gilt, dass beim Vergleich
        die Gross-Kleinschreibung ignoriert wird. Wie der Name kann auch die Adresse
        mehrere Attribut-Werte enthalten und es liegt an der Anwendung, diesen Inhalt zu
        interpretieren.
    \item[dateOfBirth] wird in Form der \texttt{GeneralizedTime} geführt. Es wird ferner
        verlangt, dass dieses Attribut maximal einmal vorkommt (\texttt{SINGLE VALUE
        TRUE}).
    \item[placeOfBirth] darf wie \texttt{dateOfBirth} maximal einmal vorkommen und nennt
        in einem unstrukturierten Text den Geburtsort des Subjektes. Der Geburtsort wird
        allerdings unter Berücksichtigung der Gross-Kleinschreibung behandelt.
    \item[gender] nennt das Geschlecht mittels "`F"', "`f"', "`M"' oder "`m"'. Dieses
        Attribut darf ebenfalls maximal einmal vorkommen.
    \item[countryOfCitizenship] zählt alle Staatsangehörigkeiten des Subjektes auf,
        folglich darf das Attribut mehrmals vorkommen. Das Land wird als 2-stelliger
        Länder-Code gemäss ISO/IEC 3166-1 ("`CH"' für Schweiz, "`DE"' für Deutschland
        etc.) hinterlegt.
    \item[countryOfResidence] nennt alle Länder der Aufenthaltsorte des Subjektes, kann
        also mehrfach vorkommen. Das Land wird ebenfalls als zweistelliger ISO Code
        hinterlegt.
    \item[pseudonym] spezifiziert ein Pseudonym des Subjektes. Neben einer ID enthält es
        das Pseudonym als Zeichenkette, welches unter Berücksichtigung der
        Gross-Kleinschreibung behandelt wird.
    \item[serialNumber] Auf dieses Attribut wird nicht eingegangen, es ist definiert in
        ISO/IEC 9594-6.
\end{description}

\subsection{Generelle Attribute}

Neben der beiden neu definierten Objekt-Klassen \texttt{pkcsEntity} und
\texttt{naturalPerson} wird im Besonderen auf einige spezifische Attribute eingegangen.
Von diesen sollen nun einige genauer betrachtet werden:

\begin{description}
    \item[contentType] Das Attribut \texttt{contentType} spezifiziert den Inhalts-Typen
        des in PKCS\#7 (oder S/MIME) signierten \texttt{ContentInfo} Objektes. In solchen
        Inhalten ist das Attribut \texttt{contentType} zwingend, falls authentifizierte
        Attribute aus PKCS\#7 vorhanden sind.
    \item[messageDigest] spezifiziert den Message Digest der Inhalte des \texttt{content}
        Feldes des \texttt{ContentInfo} Objektes. Der Message Digest wird anhand dem
        Algorithmus der signierenden Instanz berechnet. Dieses Attribut ist zwingend,
        falls authentifizierte Attribute aus PKCS\#7 Verwendung finden.
    \item[signingTime] benennt die Zeit, wann die Signatur erstellt wurde. Die Zeit wird
        gemäss ISO/IEC 9594-8 notiert, wobei Daten vor dem 1.1.1950 oder nach dem
        31.12.2049 müssen in Form der \texttt{GeneralizedTime} codiert werden, alle
        anderen Zeiten als \texttt{UTCTime}~\cite[S.12]{pkcs9}.
    \item[randomNonce] ist ein Attribut, welches es ermöglicht, gegen spezifische Attacken
        zu schützen. Beispielsweise kann ein Unterzeichner \texttt{signingTime}
        unterdrücken, um replay Attacken zu unterbinden. Das Attribut dient signierten
        Daten aus PKCS\#7 und darf nur einmal vorkommen. Das Element \texttt{RandomNonce}
        ist ein Oktett-String und muss mind. 4 Bytes lang sein.
    \item[counterSignature] erlaubt es, eine Signatur zu signieren. Das Attribut hat
        dieselbe Bedeutung wie \texttt{SignerInfo}~\cite[S.12]{pkcs7}, ausser:
        \begin{itemize}
            \item Das Feld \texttt{authenticatedAttributes} muss ein Attribut
                \texttt{messageDigest} aufweisen, falls es irgendwelche andere Attribute
                aufweist.
            \item Der Inhalt des Message Digest ist der Inhalt des \texttt{signatureValue}
                Feldes des \texttt{SignerInfo} Objektes. Das bedeutet, dass der
                signierende Prozess (welcher die Signatur signiert) den originalen Inhalt
                nicht zu kennen braucht. Zudem kann eine \texttt{counterSignature} selbst
                wieder eine \texttt{counterSignature} beinhalten, so lassen sich beliebig
                lange Ketten von \texttt{counterSignature} Objekten erstellen.
        \end{itemize}
    \item[challengePassword] spezifiziert ein Passwort, mit welchem eine Entität die
        Annullierung eines Zertifikates verlangen kann. Die Interpretation des Inhaltes
        ist wiederum der Anwendung überlassen, er wird jedoch unter Berücksichtigung von
        Gross-Kleinschreibung verglichen. Es wird bemerkt, dass der Inhalt als
        \texttt{PrintableString} encodiert werden soll, falls Internationalisierung dies
        nicht ermöglicht, sollte stattdessen \texttt{UTF8String} verwendet werden.
\end{description}


\section{PKCS 13 - Elliptische Kurven}

Die Kryptographie mit elliptischen Kurven erfährt eine zunehmende Popularität, da eine
vergleichbare Sicherheit zu etablierten Public Key Verfahren mit kleineren Schlüsseln
ermöglicht wird. Verbesserungen in der Implementierung wie der Erzeugung von elliptischen
Kurven machen das Verfahren praxistauglicher als bei seiner Einführung in den 1980-er
Jahren.

PCKS\#13 ist bis heute noch kein definitiver Standard. Der Standard ist noch immer in
Entwicklung. Der Standard soll die folgenden Aspekte der Kryptographie mit elliptischen
Kurven abdecken~\cite{pkcs13-proj}:

\begin{itemize}
    \item Parameter und Schlüssel Erzeugung und Validierung
    \item Digitale Signaturen
    \item Public Key Verschlüsselung
    \item Key Agreement (anstelle des verwundbaren anonymen Key-Exchanges wie z.Bsp.
        Diffie-Hellman)
    \item ASN.1 Syntax
    \item Überlegungen zur Sicherheit
\end{itemize}

Es sind bereits Standards in Arbeit, welche sich mit der Kryptographie mit elliptischen
Kurven befassen:
\begin{description}
    \item[ANSI X9.62] Ist ein in Entwicklung befindlicher Standard für digitale
        Signaturen.
    \item[ANSI X9.63 ] Ist ein in Entwicklung befindlicher Standard für Key Agreement.
    \item[IEEE P1363] Soll eine generelle Referenz für Public Key Verfahren verschiedener
        Techniken, inkl. elliptischer Kurven werden.
\end{description}

PKCS\#13 soll die anderen Standards vervollständigen, ein Profil der anderen Standards im
PKCS Format liefern und eine Anleitung für die Integration in andere PKCS Anwendungen (wie
beispielsweise PKCS\#7) bieten.

\subsection{Aktueller Umfang}

Zurzeit umfasst der Standard Grundlagen der folgenden Bereiche, welche allerdings keine
konkreten Implementierungen oder Standards nennt. Diese Themen können als Grundlagen der
Kryptographie mit elliptischen Kurven betrachtet werden.

\begin{description}
    \item[Functions] Grundlegende Definition einer Funktion: "`A function f from a set A
        to a set B assigns to each element a in A a unique element b in
        B."'~\cite{pkcs13-func}
    \item[Modular arithmetic] Grundlagen der modularen Arithmetik
    \item[Groups] Grundlagen der diskreten Mathematik ($\mathbb{Z}_p$ und
        $\mathbb{Z}_p^*$) in Bezug auf Gruppen.
    \item[Fields and rings] Ebenfalls mathematische Grundlagen.
    \item[Vector spaces and lattices] Behandelt die Grundlagen der linearen Algebra und
        der Vektorräume.
    \item[Boolean expressions] Die Betrachtung logischer Ausdrücke als Funktionen.
    \item[Time estimations and some complexitiy] Betrachtungen zur Komplexität.
\end{description}

Generell sind keine konkreten Vorgaben zu diesem Standard vorhanden. Zurzeit existiert
lediglich ein Vorschlag, welcher mögliche Key Agreement Schemes, Signature Schemes,
Encryption Schemes und Point Representations nennt.

\FloatBarrier
\appendix
\renewcommand{\refname}{\section{Bibliographie}}
\begin{thebibliography}{}
    \bibitem[PKCS7]{pkcs7} RSA Laboratories: PKCS\#7 Cryptographic Message Syntax Standard;
        1993
    \bibitem[PKCS9]{pkcs9} RSA Laboratories: PKCS \#9 v2.0: Selected Object Classes and
        Attribute Types; 2000
    \bibitem[PKCS13]{pkcs13} RSA Laboratories: PKCS \#13: ELLIPTIC CURVE CRYPTOGRAPHY
        STANDARD
        \url{http://www.emc.com/emc-plus/rsa-labs/standards-initiatives/pkcs-13-elliptic-curve-cryptography-standard.htm}
    \bibitem[PKCS13-proj]{pkcs13-proj} RSA Laboratories: PKCS\#13 Project overview,
        \url{http://www.emc.com/emc-plus/rsa-labs/standards-initiatives/project-overview.htm},
        1.1.2013
    \bibitem[PKCS13-func]{pkcs13-func} RSA Laboratories: A.1 Functions,
        \url{http://www.emc.com/emc-plus/rsa-labs/standards-initiatives/functions.htm},
        1.1.2013
\end{thebibliography}

\end{document}
