\documentclass[10pt,a4paper]{article}
\usepackage{ucs}
\usepackage[utf8x]{inputenc}
\usepackage[ngerman]{babel}
\selectlanguage{ngerman}
\usepackage[T1]{fontenc}

% Math stuff
\usepackage{amsmath}
\usepackage{amsfonts}
\usepackage{amssymb}

% Farben
\usepackage[usenames,x11names,dvipsnames,rgb]{xcolor}
\definecolor{grey}{rgb}{0.4,0.4,0.4}
\definecolor{lightgrey}{rgb}{0.8,0.8,0.8}
\definecolor{ultralightgrey}{rgb}{0.96,0.96,0.96}

% Grafix
\usepackage{graphicx}

% TikZ (dot2tex etc.)
\usepackage{tikz}
\usetikzlibrary{decorations, arrows, shapes}

% Farben in Tabellen
\usepackage{colortbl}

% Lange Tabellen
\usepackage{longtable}

% Gewrappte boxen (können innerhalb f{rame}box's verwendet werden)
\usepackage{minibox}

% FloatBarrier stellt sicher, dass das Literaturverzeichnis am Ende des
% Dokuments erscheint.
\usepackage{placeins}

% Hyperref
\usepackage{hyperref}
% Hypersetup
\hypersetup{
    pdftitle = {Semesterarbeit Kryptographie - Gruppenarbeit PKCS},
    pdfauthor = {Aregger Thomas, Gutknecht Jürg, Dünki Marc, Daniel David},
    pdfsubject = {Semesterarbeit Kryptographie - Gruppenarbeit PKCS},
    pdfkeywords = {Kryptographie PKCS Semesterarbeit Gruppenarbeit},
    % hidelinks
    colorlinks = true,
    linkcolor = blue,
    % urlcolor = black
    urlcolor = Blue,
    citecolor = grey
}

\urlstyle{same}

\title{Semesterarbeit Kryptographie - Gruppenarbeit  PKCS}
\author{
    Aregger Thomas \small{thomas.aregger@students.ffhs.ch}\\
    Dünki Marc \small{marc.duenki@students.ffhs.ch}\\
    Gutknecht Jürg \small{juerg.gutknecht@students.ffhs.ch}\\
    Daniel David \small{david.daniel@students.ffhs.ch}
}

\date{\today}

\begin{document}

\maketitle
\tableofcontents

\section{Einleitung}
Dieses Dokument liefert einen Überblick über das Themengebiet der Public-Key Cryptography
Standards und entstand als Teil der Semesterarbeit des Moduls Kryptologie bei Herrn Josef
Schuler.

\subsection{Gruppe}
Die Gruppe der an dieser Arbeit mitwirkenden Studenten umfasst:
\begin{itemize}
   \item Thomas Aregger
   \item David Daniel
   \item Marc Dünki (Projektleiter)
   \item Jürg Gutknecht
\end{itemize}

\subsection{Vorgehensweise}

\begin{table}[ht]
   \centering
   \begin{tabular}{|l|p{8.6cm}|} \hline
      \textbf{Datum} & \textbf{Fortschritt} \\\hline
      7.9.2013 & Gruppenbildung \\\hline
      8.9.2013 - 4.10.2013 & Individuelles Erarbeiten der Aufgabenstellung \\\hline
      4.10.2013 & Aufteilen der Themen innerhalb der Gruppe \\\hline
      4.10.2013 - 28.12.2013 & Individuelles Bearbeiten der Inhalte gemäss
         Themenverteilung \\\hline
      28.12.2013 & Besprechen des Fortschritts \\\hline
      28.12.2013 - 4.1.2014 & Individuelles Finalisieren der Inhalte gemäss
         Themenverteilung \\\hline
      4.1.2014 & Verteilen der Aufgaben zum Abschliessen der Arbeit \\\hline
      4.1.2014 - 9.1.2014 & Abschliessen der Arbeit \\\hline
         & Abgabe der Arbeit \\\hline
   \end{tabular}
   \caption{Vorgehensweise}
   \label{tab:vorgehensweise}
\end{table}

\subsection{Arbeitsteilung}

\subsubsection{Verteilung der Themen}
\begin{table}[ht]
   \centering
   \begin{tabular}{|c|l|} \hline
      \textbf{PKCS\#} & \textbf{Person} \\\hline
      1 & Thomas Aregger \\\hline
      3 & Marc Dünki \\\hline
      5 & Thomas Aregger \\\hline
      6 & Jürg Gutknecht \\\hline
      7 & David Daniel \\\hline
      8 & Thomas Aregger \\\hline
      9 & David Daniel \\\hline
      10 & Jürg Gutknecht \\\hline
      11 & Jürg Gutknecht \\\hline
      12 & Marc Dünki \\\hline
      13 & David Daniel \\\hline
      15 & Marc Dünkis \\\hline
   \end{tabular}
   \caption{Verteilung der Themen}
   \label{tab:arbeitsteilung}
\end{table}

\FloatBarrier

\subsubsection{Weitere Aufgaben}
\begin{table}[ht]
   \centering
   \begin{tabular}{|p{8cm}|l|} \hline
      \textbf{Aufgabe} & \textbf{Person} \\\hline
      Zusammenführung der Beiträge in einem Dokument & David Daniel \\\hline
      Zusammenfassung der Beiträge in einer Präsentation & \\\hline
      Abgabe der Aufgabe & \\\hline
   \end{tabular}
   \caption{Verteilung weiterer Aufgaben}
   \label{tab:weitere-aufgaben}
\end{table}

\section{Public-Key Cryptography Standards}
Die Public Key Cryptography Standards (PKCS) sind eine von RSA Laboratories~\cite{rsa-lab}
seit den 1990er- Jahren veröffentlichte Reihe an Standards zum Themengebiet der
asymmetrischen Kryptographie (Public-Key Kryptographie).

Die Gruppe der PKCS besteht aus den nachfolgend aufgelisteten Standards, welche in diesem
Kapitel detailliert betrachtet werden.

\begin{table}[ht]
    \centering
    \begin{tabular}{|c|l|} \hline
        \textbf{PKCS \#} & \textbf{Titel} \\\hline
        1 & RSA Cryptography Standard \\\hline
        3 & Diffie-Hellman Key-Agreement Standard \\\hline
        5 & Password-Based Cryptography Standard \\\hline
        6 & Extended-Certificate Syntax Standard \\\hline
        7 & Cryptographic Message Syntax Standard \\\hline
        8 & Private-Key Information Syntax Standard \\\hline
        9 & Selected Object Classes and Attribute Types \\\hline
        10 & Certification Request Syntax Standard \\\hline
        11 & Cryptographic Token Interface Standard \\\hline
        12 & Personal Information Exchange Syntax \\\hline
        13 & Elliptic Curve Cryptography Standard \\\hline
        14 & Pseudo Random Number Generation \\\hline
        15 & Cryptographic Token Information Syntax Standard \\\hline
    \end{tabular}
    \caption{Übersicht der PKCS}
    \label{tab:pkcs-overview}
\end{table}

\FloatBarrier

Die PKCS \#2 und \#4 wurden in den PKCS \#1 überführt und sind nicht mehr eigener Bestandteil
der PKCS-Familie~\cite[S.1]{kal91}.

PKCS \#14 befindet sich zur Zeit in Entwicklung, wird auf den PKCS-Seiten der RSA
Laboratories~\cite{pkcs-standards} nicht erwähnt und wird hier ebenfalls nicht genauer
betrachtet.

\subsection{PKCS \#1: RSA Cryptography Standard}

\subsection{PKCS \#3: Diffie-Hellman Key-Agreement Standard}

\subsubsection{Einleitung}
\begin{table}[ht]
    \centering
    \begin{tabular}{|l|l|} \hline
        Aktuelle Version des Standards & 1.4 \\\hline
        Datum & Revised November 1, 1993 \\\hline
        Seitenzahl & 8 \\\hline
    \end{tabular}
    % \caption{<+Caption text+>}
    % \label{tab:<+label+>}
\end{table}

\paragraph{Kurzbeschreibung RSA Laboratories}

\begin{quotation}
    \textit{This standard describes a method for implementing Diffie-Hellman key
    agreement. The intended application of this standard is in protocols for establishing
    secure communications.}
\end{quotation}

\subsubsection{Zusammenfassung}

\paragraph{Scope}
Das Diffie-Hellmann-Schlüsselaustausch Protokoll wird in der Kryptografie verwendet, um
zwischen zwei Kommunikationspartner einen geheimen Schlüssel zu erzeugen. Dieser wird dann
vornehmlich als geheimer Schlüssel für synchrone Verschlüsselungsmechanismen verwendet.
Mit diesem Protokoll soll eine grosse Problematik der synchronen Verschlüsselung, nämlich
der des Austauschs des geheimen Schlüssels, gelöst werden. Da es sich beim Diffie-Hellann
um ein Protokoll handelt, welches sich auf das Problem des diskreten Logarithmus stützt,
kann man für den Einsatz zum Beispiel Elliptische Kurven (ECC) verwenden.

\paragraph{Sicherheit}
Beide Kommunikationspartner senden sich eine Nachricht über ein nicht gesichertes Netz zu
(Internet). Falls nun ein Angreifer diese beiden Nachrichten abfängt und aus diesen beiden
Nachrichten den geheimen Schlüssel generieren möchte, muss er das Diffie-Hellmann-Problem
lösen. Von diesem wird jedoch angenommen, dass es praktisch unlösbar ist. Allerdings,
sobald ein Angreifer diese Nachrichten verändern kann (Man-in-the-Middle-Attack), ist der
Diffie- Hellmann-Schlüsselaustausch nicht mehr sicher.  Das impliziert, dass dieses
Protokoll mit Mechanismen der Authentiziät kombiniert werden sollte, welche dann
sicherstellen, dass Nachrichten nicht umbemerkt verändert werden können. Alternativ kann
auch ein Zero-Knowledge- Beweis zum Einsatz kommen, wie z.B. das Fiat-Shamir-Verfahren.
Dieses Verfahren ermöglicht es einem Kommunikationspartner zu beweisen, dass er das
Geheimnis kennt, ohne es aber zu verraten.

\paragraph{Ablauf Schlüsselerzeugung}

\subparagraph{Generierung der Parameter}
Eine zentrale Authorität wählt eine ungerade Primzahl "`p"'. Ebenfalls wird eine
Primitivwurzel "`g"' gewählt, welche erfüllt dass, 0 < g < p gilt. Bei einigen Methoden
hängt der Aufwand für das berechnen des diskreten Logarithmus vo der länge der Primzahl
ab. Für andere ist die länge der privaten Geheimzahl ausschlaggebend. Deshalb kann gerade
für dieses Verfahren, durch die zentrale Autorität eine maximale Länge der privaten
Geheimzahl vorgegeben werden. Dies ermöglicht die Zeit für das Berechnen relativ klein zu
halten, während trotzdem ein gewisses Niveau von Sicherheit gewährleistet werden kann.

\subparagraph{Vorgehen zur Schlüsselerzeugung}
Der Ablauf der Schlüsselerzeugen teilt sich in zwei Phasen ein, wobei beide Phasen in drei
Schritten ablaufen. Beide Kommunikationspartner durchlaufen diese beiden Phasen gleichzeitig.
Für die erste Phase nehmen beide Kommunikationspartner die öffentlich bekannten Zahlen
"`p"' (Primzahl) und "`g"' (Primitivwurzel) als Input.

\subparagraph{Phase 1}
\begin{description}
    \item[Schritt 1] Erzeugen der privaten Geheimzahl
        \begin{itemize}
            \item Es wird eine natürliche Zahl "`x"' gewählt, so dass $0 < x < p - 1$ gilt
        \end{itemize}
    \item[Schritt 2] Potenzieren
        \begin{itemize}
            \item Es wird ein öffentlicher int Wert "`y"' errechnet mit: $y = g^x \mod
                p$, $0 < y < p$
        \end{itemize}
    \item[Schritt 3] Konvertierung von Integer-to-octet-string
        \begin{itemize}
            \item Der öffentliche int Wert "`y"' wird in ein octet-string PV (Public
                Value) mit Länge "`k"' gewandelt mit $y = \sum_{i=1}^k 2^{8(k-i) PV_i}$
        \end{itemize}
\end{description}

Nach der ersten Phase tauschen die beiden Kommunikationspartner ihren öffentlichen Wert
(PV) aus, welcher soeben errechnet wurde. Für die zweite Phase nehmen beide
Kommunikationspartner den soeben erhaltenen Public Value PV' des Kommunikationspartners,
wie auch die eigene private Geheimzahl als Input.

\subparagraph{Phase 2}
\begin{description}
    \item[Schritt 1] Konvertierung von Octet-string-to-integer
        \begin{itemize}
            \item Der öffentliche octet-string PV' des Kommunikationspartners wird
                umgewandelt in ein int-Wert "`y"' mit: $y = \sum_{i=1}^k 2^{8(k-i) PV_i'}$
        \end{itemize}
    \item[Schritt 2] Potenzieren
        \begin{itemize}
            \item Die errechnete Zahl "`y"' wird nun mit der eigentlichen Geheimzahl "`x"'
                potenziert und mod "`p"' gerechnet: $Z = (y')^x \mod p$, $0 < z < p$.
        \end{itemize}
        Die daraus erzeugte Zahl "`z"' ist nur der gemeinsame Geheimschlüssel
    \item[Schritt 3] Konvertierung von Integer-to-octet-string
        \begin{itemize}
            \item Der int-Wert "`z"' wird nun noch in einen octet-string SK(Secret Key)
                umgewandelt: $z = \sum_{i=1}^k 2^{8(k-i)SK_i}$
        \end{itemize}
\end{description}

Der in der zweiten Phase von beiden erzeugte Octet-string, auch bekannt als secret key
(SK), kann nun von beiden für das Verschlüsseln von Nachrichten angewendet werden.

\subsubsection{Vergleich zur Vorlesung Krypt}
Die im Vorgehen mehrfach vollzogene Umwandlung von Integer in String und umgekehrt, wurde
in unseren Vorlesungen nicht betrachtet. Dieser Schritt ist jedoch für die Theorie nicht
zentral und lediglich wichtig für eine konkrete Implementierung in der Praxis von DH.
Zudem werden in diesem PKCS die Domain Werte, also Primzahl p sowie eine Primitivwurzel g,
nicht von den beiden Kommunikationsparteien sondern von einer zentralen Authorität
gewählt.

\subsection{PKCS 7 - Standard zur Syntax kryptographischer Nachrichten}

\begin{description}
    \item[PKCS \#7] definiert eine generelle Syntax zur Beschreibung von Inhalten welche in
        Verbindung mit kryptographischen Verfahren stehen können, zum Beispiel Signaturen.
        Die Syntax erlaubt Rekursion, so dass beispielsweise Inhalte signiert werden
        können, welche zuvor von einer anderen Instanz signiert wurden etc.
\end{description}

Folgendes sind Beispiele von Anwendungen, welche dieser Standard adressiert:
\begin{itemize}
    \item Signieren von digitalen Nachrichten
    \item Digest (hash) von digitalen Nachrichten
    \item Authentisierung von Nachrichten (MAC)
    \item Verschlüsselung digitaler Inhalte
\end{itemize}

\subsubsection{Aufbau}

Es werden generell zwischen zwei Klassen von Inhalts-Typen unterschieden:
\begin{description}
    \item[base] data enthält "`plain data"', also Daten, welche keine kryptographischen
        Erweiterungen ("`enhancements"') aufweisen.
    \item[enhanced] data enthält Inhalt eines bestimmten Typs (evtl. verschlüsselt) und
        weitere kryptographische Erweiterungen.
\end{description}

Insgesamt werden vom Standard sechs Inhalts-Typen definiert, weitere könnten in der
Zukunft hinzu kommen:
\label{list:content-types}
\begin{itemize}
    \item data
    \item signed data
    \item enveloped data
    \item signed-and-enveloped data
    \item digested data
    \item encrypted data
\end{itemize}

Nur "`data"' gehört zur Klasse "`base"', alle weiteren Typen gehören zur Kategorie
"`enhanced"'. Da die Typen der enhanced Klasse selbst Inhalte anderer Typen beinhalten,
wird auch von dem sogenannten "`inner"' und "`outer"' content gesprochen. Der "`inner"'
content ist somit der vom "`outer"' content erweiterte Inhalt.

\subsubsection{Der Inhalt}

Der Standard exportiert schliesslich einen Typen \texttt{ContentInfo}, welcher den
entsprechenden Inhalt zu repräsentieren vermag. Eine Nachricht, ein Element resp. ein
Objekt dieses Standards weist die folgende Syntax auf:

\begin{verbatim}
ContentInfo ::= SEQUENCE {
    contentType ContentType,
    content
        [0] EXPLICIT ANY DEFINED BY contentType OPTIONAL
}
\end{verbatim}

\begin{description}
    \item[contentType] benennt den Typ des Inhaltes. Es handelt sich um einen Object
        identifier, welcher den Inhalts-Typen gemäss obiger Liste~\ref{list:content-types}
        (data, signed data etc.) enthält. Er weist die folgende Syntax auf:
        \begin{verbatim}
        ContentType ::= OBJECT IDENTIFIER
        \end{verbatim}
    \item[content] ist der Inhalt der Nachricht. Das Feld ist optional und falls es nicht
        enthalten ist, muss der gewünschte Inhalt anderweitig zur Verfügung gestellt
        werden (wird mittels \texttt{contentType} kommuniziert).
\end{description}

\subsubsection{Die Inhalts-Typen}

\paragraph{Data}
Der data content type ist ein arbiträrer octet string, welcher generell keine interne
Struktur aufweist (jedoch aufweisen kann) und in Form einer ASCII Zeichenketten betrachtet
wird. Die genaue Interpretation des Inhaltes wird der Anwendung überlassen.

\paragraph{Signed-data}
Der signed-data content type besteht aus dem signierten Inhalt eines beliebigen Typs sowie
verschlüsselter Digests des Inhaltes, welche von einer beliebigen Anzahl Instanzen
signiert wurden.

Die Syntax des signet-data content type ist folgendermassen gegeben \cite[S.10]{pkcs7}:
\begin{verbatim}
SignedData ::= SEQUENCE {
    version Version,
    digestAlgorithms DigestAlgorithmIdentifiers,
    contentInfo ContentInfo,
    certificates [0] IMPLICIT ExtendedCertificatesAndCertificates
        OPTIONAL,
    crls [1] IMPLICIT CertificateRevocationLists OPTIONAL,
    signerInfos SignerInfos
}
DigestAlgorithmIdentifiers ::= SET OF DigestAlgorithmIdentifier
SignerInfos ::= SET OF SignerInfo
\end{verbatim}

Der Prozess, wie signierter Inhalt erzeugt wird, wird folgendermassen festgehalten
\cite[S.10]{pkcs7}:
\begin{enumerate}
    \item Pro jede signierende Instanz wird der Message Digest des Inhaltes gemäss dem
        Algorithmus der signierenden Instanz erstellt.
    \item Pro signierende Instanz wird mit dessen privatem Schlüssel jeder Message Digest
        dieser Instanz und dessen zugehörigen Angaben verschlüsselt.
    \item Pro signierende Instanz werden der verschlüsselte Message Digest und andere
        Instanz-spezifische Informationen in einem \texttt{SignerInfo} Objekt abgelegt.
        Die Zertifikate und certificate-revocation Listen werden in diesem Schritt
        ermittelt.
    \item Sämtliche Message-Digest Algorithmen und \texttt{SignerInfo} Objekte für alle
        signierenden Instanzen werden mit dem Inhalt im \texttt{SignedDataValue} Objekt
        abgelegt.
\end{enumerate}

\paragraph{Enveloped}
Der enveloped-data content type beinhaltet verschlüsselte Daten sowie die verschlüsselten
Schlüssel für eine beliebige Anzahl Empfänger, womit der Inhalt wieder entschlüsselt
werden kann. Die Idee ist ein sogenanntes "`Digital Envelope"', dadurch können die
Vorteile des Public Key Algorithmus mit den Vorzügen der symmetrischen Verschlüsselung
genutzt werden (hybrid).

Das Element enthält neben einer Version und den verschlüsselten Inhalten eine Menge von
Empfänger Angaben. Darin werden der verschlüsselte Schlüssel sowie die Angaben über den
zugrundeliegenden Algorithmus hinterlegt:
\begin{verbatim}
EnvelopedData ::= SEQUENCE {
    version Version,
    recipientInfos RecipientInfos,
    encryptedContentInfo EncryptedContentInfo
}
RecipientInfos ::= SET OF RecipientInfo
EncryptedContentInfo ::= SEQUENCE {
    contentType ContentType,
    contentEncryptionAlgorithm
    ContentEncryptionAlgorithmIdentifier,
    encryptedContent
    [0] IMPLICIT EncryptedContent OPTIONAL
}
EncryptedContent ::= OCTET STRING
\end{verbatim}
\label{syntax:enveloped}

\paragraph{Signed and enveloped}
Der signed and enveloped content type kombiniert quasi die signed und die enveloped Typen.
Im Besonderen wird der genaue Prozess zur Erstellung des entsprechenden Inhaltes genannt
\cite[S.22]{pkcs7} und der genaue Aufbau des Typs beschrieben \cite[S.23]{pkcs7}.

\paragraph{Digested}
Der digested content type beinhaltet einen Inhalt beliebigen Typs sowie einen Message
Digest dazu. Dies dient grundsätzlich dazu, die Integrität des Inhaltes zu gewährleisten.
Dieser Inhalt wird daher typischerweise in einen enveloped content type integriert.

\paragraph{Encrypted}
Der encrypted content type Inhalt enthält selbst keine Empfänger oder Schlüssel. Der Typ
ist eher dazu gedacht, für lokale Verschlüsselung verwendet zu werden. Der Typ beinhaltet
lediglich eine Version und den verschlüsselten Inhalt wie er Bestandteil des enveloped
content type ist (Syntax \ref{syntax:enveloped}, \texttt{EncryptedContentInfo}).

\subsection{PKCS 9 - Ausgewählte Objektklassen und Attribute}
In PKCS\#9 werden einige Objektklassen und Attribute behandelt, welche Bestandteile
anderer PKCS Dokumente sind und von unterschiedlichen Dokumenten gleichermassen
referenziert werden.

Der Standard führt die folgenden zwei neuen Objekt-Klassen und deren zugehörigen Attribute
ein:
\begin{itemize}
    \item \texttt{pkcsEntity}
    \item \texttt{naturalPerson}
\end{itemize}

\subsubsection{pkcsEntity}
Die \texttt{pcksEntity} Objekt-Klasse ist dazu gedacht, Attribute von beliebigen PKCS
Entitäten zu beherbergen. Sie wurde für die Verwendung von LDAP basierten
Verzeichnis-Diensten entwickelt.

Die Syntax ist folgendermassen gegeben: 
\begin{verbatim}
pkcsEntity OBJECT-CLASS ::= {
    SUBCLASS OF { top }
    KIND auxiliary
    MAY CONTAIN { PKCSEntityAttributeSet }
    ID pkcs-9-oc-pkcsEntity
}
\end{verbatim}

Die Klasse sieht eine ID sowie ein optionales Attribut vor. Die folgenden Attribute werden
allesamt dafür verwendet, die zugehörigen Informationen in einem Verzeichnis-Dienst
abzulegen.
\begin{description}
    \item[pKCS7PDU] Die in PKCS\#7 definierten geschützten Daten (enveloped, signed etc.)
        werden mit diesem Attribut verwendet.
    \item[userPKCS12] In PKCS\#12 wird ein Format für den Austausch von Angaben über die
        persönliche Identität definiert. Dieses Attribut kann hierfür verwendet werden.
    \item[pKCS15Token] In PKCS\#15 wird ein Format für kryptographische Tokens definiert,
        welche in diesem Attribut abgelegt werden können.
    \item[encryptedPrivateKeyInfo] PCKS\#8 definiert ein Format für verschlüsselte private
        Schlüssel, welche in diesem Attribut enthalten sein können.
\end{description}

\subsubsection{naturalPerson}
Die Objekt-Klasse \texttt{naturalPerson} wurde wie die Klasse \texttt{pkcsEntity} für die
Verwendung in Verzeichnis-Diensten erstellt. \texttt{naturalPerson} ist dafür gedacht,
Attribute von natürlichen Personen (Menschen) zu beherbergen.

Die Syntax und der Aufbau ähneln stark derjenigen der \texttt{pkcsEntity}:
\begin{verbatim}
naturalPerson OBJECT-CLASS ::= {
    SUBCLASS OF { top }
    KIND auxilary
    MAY CONTAIN { NaturalPersonAttributeSet }
    ID pkcs-9-oc-naturalPerson
}
\end{verbatim}

Für \texttt{naturalPerson} sind die folgenden Attribute definiert:
\begin{description}
    \item[emailAddress] spezifiziert eine oder mehrere E-Mail Adressen in Form einer
        unstrukturierten ASCII Zeichenkette. Es liegt an der Anwendung, die Adressen zu
        interpretieren. Speziell an diesem Element ist die \texttt{EQUALITY MATCHING RULE}
        als \texttt{pkcs9CaseIgnoreMatch}, welche besagt, dass falls zwei E-Mail Adressen
        miteinander verglichen werden, wird die Gross-Kleinschreibung ignoriert.
    \item[unstructuredName] spezifiziert den oder die Namen eines Subjektes als
        unstrukturierte ASCII Zeichenkette. Ein unstrukturierter Name kann mehrere
        Attribut-Werte enthalten, es liegt auch hier an der Anwendung, den Namen zu
        interpretieren. Wie bei der E-Mail Adresse wird beim Vergleich zweier
        unstrukturierter Adressen die Gross-Kleinschreibung ignoriert.
    \item[unstructuredAddress] nennt die Adresse des Subjektes. Auch hierbei handelt es
        sich um vollkommen unstrukturierten Inhalt. Auch hier gilt, dass beim Vergleich
        die Gross-Kleinschreibung ignoriert wird. Wie der Name kann auch die Adresse
        mehrere Attribut-Werte enthalten und es liegt an der Anwendung, diesen Inhalt zu
        interpretieren.
    \item[dateOfBirth] wird in Form der \texttt{GeneralizedTime} geführt. Es wird ferner
        verlangt, dass dieses Attribut maximal einmal vorkommt (\texttt{SINGLE VALUE
        TRUE}).
    \item[placeOfBirth] darf wie \texttt{dateOfBirth} maximal einmal vorkommen und nennt
        in einem unstrukturierten Text den Geburtsort des Subjektes. Der Geburtsort wird
        allerdings unter Berücksichtigung der Gross-Kleinschreibung behandelt.
    \item[gender] nennt das Geschlecht mittels "`F"', "`f"', "`M"' oder "`m"'. Dieses
        Attribut darf ebenfalls maximal einmal vorkommen.
    \item[countryOfCitizenship] zählt alle Staatsangehörigkeiten des Subjektes auf,
        folglich darf das Attribut mehrmals vorkommen. Das Land wird als 2-stelliger
        Länder-Code gemäss ISO/IEC 3166-1 ("`CH"' für Schweiz, "`DE"' für Deutschland
        etc.) hinterlegt.
    \item[countryOfResidence] nennt alle Länder der Aufenthaltsorte des Subjektes, kann
        also mehrfach vorkommen. Das Land wird ebenfalls als zweistelliger ISO Code
        hinterlegt.
    \item[pseudonym] spezifiziert ein Pseudonym des Subjektes. Neben einer ID enthält es
        das Pseudonym als Zeichenkette, welches unter Berücksichtigung der
        Gross-Kleinschreibung behandelt wird.
    \item[serialNumber] Auf dieses Attribut wird nicht eingegangen, es ist definiert in
        ISO/IEC 9594-6.
\end{description}

\subsubsection{Generelle Attribute}

Neben der beiden neu definierten Objekt-Klassen \texttt{pkcsEntity} und
\texttt{naturalPerson} wird im Besonderen auf einige spezifische Attribute eingegangen.
Von diesen sollen nun einige genauer betrachtet werden:

\begin{description}
    \item[contentType] Das Attribut \texttt{contentType} spezifiziert den Inhalts-Typen
        des in PKCS\#7 (oder S/MIME) signierten \texttt{ContentInfo} Objektes. In solchen
        Inhalten ist das Attribut \texttt{contentType} zwingend, falls authentifizierte
        Attribute aus PKCS\#7 vorhanden sind.
    \item[messageDigest] spezifiziert den Message Digest der Inhalte des \texttt{content}
        Feldes des \texttt{ContentInfo} Objektes. Der Message Digest wird anhand dem
        Algorithmus der signierenden Instanz berechnet. Dieses Attribut ist zwingend,
        falls authentifizierte Attribute aus PKCS\#7 Verwendung finden.
    \item[signingTime] benennt die Zeit, wann die Signatur erstellt wurde. Die Zeit wird
        gemäss ISO/IEC 9594-8 notiert, wobei Daten vor dem 1.1.1950 oder nach dem
        31.12.2049 müssen in Form der \texttt{GeneralizedTime} codiert werden, alle
        anderen Zeiten als \texttt{UTCTime}~\cite[S.12]{pkcs9}.
    \item[randomNonce] ist ein Attribut, welches es ermöglicht, gegen spezifische Attacken
        zu schützen. Beispielsweise kann ein Unterzeichner \texttt{signingTime}
        unterdrücken, um replay Attacken zu unterbinden. Das Attribut dient signierten
        Daten aus PKCS\#7 und darf nur einmal vorkommen. Das Element \texttt{RandomNonce}
        ist ein Oktett-String und muss mind. 4 Bytes lang sein.
    \item[counterSignature] erlaubt es, eine Signatur zu signieren. Das Attribut hat
        dieselbe Bedeutung wie \texttt{SignerInfo}~\cite[S.12]{pkcs7}, ausser:
        \begin{itemize}
            \item Das Feld \texttt{authenticatedAttributes} muss ein Attribut
                \texttt{messageDigest} aufweisen, falls es irgendwelche andere Attribute
                aufweist.
            \item Der Inhalt des Message Digest ist der Inhalt des \texttt{signatureValue}
                Feldes des \texttt{SignerInfo} Objektes. Das bedeutet, dass der
                signierende Prozess (welcher die Signatur signiert) den originalen Inhalt
                nicht zu kennen braucht. Zudem kann eine \texttt{counterSignature} selbst
                wieder eine \texttt{counterSignature} beinhalten, so lassen sich beliebig
                lange Ketten von \texttt{counterSignature} Objekten erstellen.
        \end{itemize}
    \item[challengePassword] spezifiziert ein Passwort, mit welchem eine Entität die
        Annullierung eines Zertifikates verlangen kann. Die Interpretation des Inhaltes
        ist wiederum der Anwendung überlassen, er wird jedoch unter Berücksichtigung von
        Gross-Kleinschreibung verglichen. Es wird bemerkt, dass der Inhalt als
        \texttt{PrintableString} encodiert werden soll, falls Internationalisierung dies
        nicht ermöglicht, sollte stattdessen \texttt{UTF8String} verwendet werden.
\end{description}


\subsection{PKCS 13 - Elliptische Kurven}

Die Kryptographie mit elliptischen Kurven erfährt eine zunehmende Popularität, da eine
vergleichbare Sicherheit zu etablierten Public Key Verfahren mit kleineren Schlüsseln
ermöglicht wird. Verbesserungen in der Implementierung wie der Erzeugung von elliptischen
Kurven machen das Verfahren praxistauglicher als bei seiner Einführung in den 1980-er
Jahren.

PCKS\#13 ist bis heute noch kein definitiver Standard. Der Standard ist noch immer in
Entwicklung. Der Standard soll die folgenden Aspekte der Kryptographie mit elliptischen
Kurven abdecken~\cite{pkcs13-proj}:

\begin{itemize}
    \item Parameter und Schlüssel Erzeugung und Validierung
    \item Digitale Signaturen
    \item Public Key Verschlüsselung
    \item Key Agreement (anstelle des verwundbaren anonymen Key-Exchanges wie z.Bsp.
        Diffie-Hellman)
    \item ASN.1 Syntax
    \item Überlegungen zur Sicherheit
\end{itemize}

Es sind bereits Standards in Arbeit, welche sich mit der Kryptographie mit elliptischen
Kurven befassen:
\begin{description}
    \item[ANSI X9.62] Ist ein in Entwicklung befindlicher Standard für digitale
        Signaturen.
    \item[ANSI X9.63 ] Ist ein in Entwicklung befindlicher Standard für Key Agreement.
    \item[IEEE P1363] Soll eine generelle Referenz für Public Key Verfahren verschiedener
        Techniken, inkl. elliptischer Kurven werden.
\end{description}

PKCS\#13 soll die anderen Standards vervollständigen, ein Profil der anderen Standards im
PKCS Format liefern und eine Anleitung für die Integration in andere PKCS Anwendungen (wie
beispielsweise PKCS\#7) bieten.

\subsubsection{Aktueller Umfang}

Zurzeit umfasst der Standard Grundlagen der folgenden Bereiche, welche allerdings keine
konkreten Implementierungen oder Standards nennt. Diese Themen können als Grundlagen der
Kryptographie mit elliptischen Kurven betrachtet werden.

\begin{description}
    \item[Functions] Grundlegende Definition einer Funktion: "`A function f from a set A
        to a set B assigns to each element a in A a unique element b in
        B."'~\cite{pkcs13-func}
    \item[Modular arithmetic] Grundlagen der modularen Arithmetik
    \item[Groups] Grundlagen der diskreten Mathematik ($\mathbb{Z}_p$ und
        $\mathbb{Z}_p^*$) in Bezug auf Gruppen.
    \item[Fields and rings] Ebenfalls mathematische Grundlagen.
    \item[Vector spaces and lattices] Behandelt die Grundlagen der linearen Algebra und
        der Vektorräume.
    \item[Boolean expressions] Die Betrachtung logischer Ausdrücke als Funktionen.
    \item[Time estimations and some complexitiy] Betrachtungen zur Komplexität.
\end{description}

Generell sind keine konkreten Vorgaben zu diesem Standard vorhanden. Zurzeit existiert
lediglich ein Vorschlag, welcher mögliche Key Agreement Schemes, Signature Schemes,
Encryption Schemes und Point Representations nennt.

\FloatBarrier
\appendix
\renewcommand{\refname}{\section{Quellen}}
\begin{thebibliography}{}
    \bibitem[RSA-Lab]{rsa-lab} RSA Laboratories:
        \url{http://www.emc.com/domains/rsa/index.htm}, 1.1.2014
    \bibitem[KAL91]{kal91} Kalinski \& Burton S.: An Overview of the PKCS Standards, 1991
    \bibitem[PKCS-Standards]{pkcs-standards} RSA Laboratories: PUBLIC-KEY CRYPTOGRAPHY
        STANDARDS,
        \url{http://www.emc.com/emc-plus/rsa-labs/standards-initiatives/public-key-cryptography-standards.htm},
        1.1.2014
    \bibitem[PKCS7]{pkcs7} RSA Laboratories: PKCS\#7 Cryptographic Message Syntax Standard;
        1993
    \bibitem[PKCS9]{pkcs9} RSA Laboratories: PKCS \#9 v2.0: Selected Object Classes and
        Attribute Types; 2000
    \bibitem[PKCS13]{pkcs13} RSA Laboratories: PKCS \#13: ELLIPTIC CURVE CRYPTOGRAPHY
        STANDARD
        \url{http://www.emc.com/emc-plus/rsa-labs/standards-initiatives/pkcs-13-elliptic-curve-cryptography-standard.htm}
    \bibitem[PKCS13-proj]{pkcs13-proj} RSA Laboratories: PKCS\#13 Project overview,
        \url{http://www.emc.com/emc-plus/rsa-labs/standards-initiatives/project-overview.htm},
        1.1.2013
    \bibitem[PKCS13-func]{pkcs13-func} RSA Laboratories: A.1 Functions,
        \url{http://www.emc.com/emc-plus/rsa-labs/standards-initiatives/functions.htm},
        1.1.2013
\end{thebibliography}

\end{document}
